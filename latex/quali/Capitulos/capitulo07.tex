\chapter{Aprendizado de regras}

A proteína Ga98 e seus mutantes, os quais sofrem alterações globais na estrutura secundária, são casos interessantes para o teste de novas metodologias de predição de estrutura secundária. Nas metodologias atuais, que comumente utilizam redes neurais, a predição é feita utilizando uma janela de resíduos, em geral com comprimentos de 9, 11 ou 13 resíduos, onde o resíduo central da janela é classificado pela rede neural. Como a predição nas demais janelas presentes na sequência polipeptídica não influencia na classificação da janela, o método apresenta a limitação de responder apenas localmente às variações dos dados de entrada.  

Por outro lado, os autômatos celulares, apesar de evoluirem de acordo com regras locais, tem a capacidade de propagar as variações locais e influenciar o surgimento ou alteração de padrões globais, distantes do ponto de origem da variação. 

Para avaliar a capacidade dos modelos propostos e da eficácia do método de otimização em encontrar regras capazes de reproduzir o padrão correspondente às estruturas secundárias, testamos a nossa metodologia nessas quatro proteínas.



\chapter{Aprendizado das regras gerais}