\chapter{Aprendizado de regras}

A proteína Ga98 e seus mutantes, os quais sofrem alterações globais na estrutura secundária, são casos interessantes para o teste de novas metodologias de predição de estrutura secundária. Nas metodologias atuais, que comumente utilizam redes neurais, a predição é feita utilizando uma janela de resíduos, em geral com comprimentos de 9, 11 ou 13 resíduos, onde o resíduo central da janela é classificado pela rede neural. Como a predição nas demais janelas presentes na sequência polipeptídica não influencia na classificação da janela, o método apresenta a limitação de responder apenas localmente às variações dos dados de entrada.  

Por outro lado, os autômatos celulares, apesar de evoluirem de acordo com regras locais, tem a capacidade de propagar as variações locais e influenciar o surgimento ou alteração de padrões globais, distantes do ponto de origem da variação. 

Para avaliar a capacidade dos modelos propostos e da eficácia do método de otimização em encontrar regras capazes de reproduzir o padrão correspondente às estruturas secundárias, testamos a nossa metodologia nessas quatro proteínas.



\chapter{Aprendizado das regras gerais}




\section{contagem de aa}
CCW      2
CMW      2
WMC      2
WPC      3
WCW      3
CHW      4
MWC      4
CCC      4
CMC      4
WCC      4
CWW      4
WWM      5
MCW      5
CWC      5
MWM      6
CCM      6
MWH      6
CWH      6
FWC      6
QCC      6
QCW      6
WWW      6
WCH      7
CMM      7
YCW      7
WCM      7
CIW      7
HWC      7
CQW      7
CYW      8
..     ...
AAE   1000
LKE   1001
AEA   1011
EAA   1021
AGA   1029
AGL   1032
AAV   1034
AEL   1046
ELA   1051
EEL   1066
ALG   1076
LGL   1080
LAG   1082
VAA   1105
ALE   1111
LEA   1111
AVA   1118
ELL   1129
LAL   1131
LLE   1166
LAE   1172
AAG   1246
ALL   1289
LLA   1296
EAL   1316
LAA   1478
AAL   1534
ALA   1563
AAA   1682
HHH   6143