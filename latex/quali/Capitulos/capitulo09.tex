\chapter{Perspectivas futuras}

Neste capítulo discutiremos alterações que estão sendo planejadas para melhorar o método que estamos desenvolvendo. Essas alterações foram pensadas baseando-se nos resultados obtidos até o momento e nas possibilidades para melhorá-los.

\subsection{Função de fitness}

Os resultados obtidos até o momento indicam que a função de fitness precisa de modificada para cornseguirmos melhorar a acurácia da predição. A função utilizada até o momento não mostrou-se eficaz ao lidar com classes desbalanceadas como podemos notar pela menor acurácia obtida em fitas quando comparada a hélices e coils.

A ineficácia ao lidar com classes desbalanceadas, não prejudica somente a predição das classes menos numerosas, mas também a das classes mais numerosas, uma vez que essas tendem a ser superestimadas.

Entre as funções de fitness que planejamos testar estão a CBA e a MCC.  

%Descrever ambas as funçoes. 



\subsection{CA probabilístico durante a otimização das regras pelo EDA}

Como descrito anteriormente, durante a otimização, o EDA envia probabilidades para os nós de trabalho que irão gerar regras de transição candidatas que finalmente disputarão os torneios. Atualmente, estas regras de transição geradas nos nós de trabalho são determinísticas. 
 
Uma alternativa que planejamos implementar na continuidade deste trabalho é a modificação dos autômatos celulares utilizados durante a otimização das regras para autômatos celulares do tipo probabilísticos. Com isso, as probabilidades do EDA não representariam apenas as probabilidades das transições na população do EDA, mas sim a probabilidade de transição do elemento em si. Acreditamos que essa modifificação aproximaria as regras de transição de um modelo físico, onde as probabilidades de transição teriam relação com a energia livre de uma transição da trinca de estados. 



% Utilizar um CA probabilistico durante a fase de otimização das regras através do EDA.

% Para alterar as probabilidades, será necessário que probabilidades sofram variação durante a criação de um indivíduo. A variação não devera ser brusca (distribuiçao normal), para manter uma exploração local da regiao. Essa variação podera ser dependente de outros fatores, por exemplo, da propria variação de probabilidades ao longo da evolução do EDA 

\subsection{Predição de estados conformacionais dos resíduos}

Uma possibilidade que está sendo avaliada é modificação do objetivo da predição. A predição de elementos da estrutura secundária por resíduo não apresenta uma correspondente física uma vez que um resíduo raramente compõe isoladamente uma estrutura secundária. Por exemplo, hélices e fitas são formadas pela repetição de resíduos que se encontram em estados conformacionais específicos. Logo, um resíduo isolado no estado conformacional de uma hélice, poderia fazer fazer parte outro elemento de estrutura secundária como um coil ou volta.

Por sua vez, uma estrutura secundária poderia ser classificada pelo estado conformacional dos resíduos, sendo o estado conformacional representado por regiões de ângulos phi e psi. Assim, ao invés do autômato celular predizer os elementos de estrutura secundária ele poderia predizer estados conformacionais referentes a regiões dos ângulos. 

Esta representação, combinada com o uso de um CA probabilístico, representaria as probabilidades dos estados conformacionais, fornecendo também as alterações de probabilidades que ocorrem ao longo da evolução do autômato celular. Assim, a evolução desse CA probabilístico resultaria na predição das probabilidades de cada aminoácido estar em cada estado conformacional.

Isso traria a vantagem de, além de ser possível descrever os elementos de estrutura secundária utilizando os estados conformacionais de cada resíduo, seria possível obter também informações da estrutura tridimensional das proteínas. Mesmo que essa abordagem diminuísse a acurácia na predição de elementos de estrutura secundária, o ganho de informção que poderíamos na conformação tridimensional, por exemplo, em coils, poderia compensar a mudança para a adoção dos estados conformacionais na predição.

Por outro lado, o número de estados conformacionais depende da forma como regiões de ângulos phi e psi serão discretizadas uma vez que o CA exige estados discretos. A princípio, poderíamos optar por poucos estados como a região de hélices, região de fitas e outras regiões, sendo essa última, todas as não hélices e não fitas. A princípio, um número maior de estados conformacionais poderia ser usado permitindo identificar, por exemplo, regiões de volta. No entando, isso aumentaria exponencialmente o tamanho da regra.


% Utilizar regiões de ângulos torcionais ao invés de estrutura secundária

%\subsection{Utilização de informação evolutiva}

% Utilizar as relações evolutivas entre as trincas (conservação e variabilidade das trincas) para criar relacoes entre as probabilidades de transição.

% \chapter{Alternativas em análise}